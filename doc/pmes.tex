\documentclass[a4paper,10pt]{article}
\usepackage[utf8]{inputenc}

%opening
\title{PMES}
\author{Workflows and distributed computing}

\begin{document}

\maketitle

\section{Configuration}
\subsection{User configuration}
PMES vm should have the user pmes.

\subsection{Folder structure}
At pmes home directory (/home/pmes/) should be the following folders:
\begin{itemize}
\item pmes
 \subitem Dashboard
 \subitem logs
 \subitem config
 \subitem jobs
\end{itemize}

tomcat7 user should have permission to write on folders logs and jobs.
Give permissions to the tomcat7 user:
\begin{verbatim}
 sudo usermod -a -G tomcat7 pmes
 sudo chmod g+w myfolder
\end{verbatim}

\subsection{config file}
The config file should be at \texttt{/home/pmes/pmes/config/config.xml}. The structure is as follows:

\begin{verbatim}
 <pmes>
    <workspace>/home/pmes/pmes</workspace>
    <connector className="rOCCIHelper">
        <property>
            <key>providerName</key>
            <value>ONE</value>
        </property>
        ...              
    </connector>    
    <hosts>
        <host>
            <name>host12</name>
            <MAX_CPU>2400</MAX_CPU>
            <MAX_MEM>99195808</MAX_MEM>
        </host>
        ...
    </hosts>
    <logPath>/home/pmes/pmes/logs</logPath>
    <logLevel>DEBUG</logLevel>
    <timeout>60</timeout>
    <pollingInterval>5</pollingInterval>
    <runCmd>
        <cmd>echo "config script"</cmd>
    </runCmd>    
    <auth-keys>
        <key>...</key>
    </auth-keys>
</pmes>
\end{verbatim}

\subsection{ssh}
Disable known\_hosts
\begin{verbatim}
Add "StrictHostKeyChecking no" to /etc/ssh/ssh_config
cd ~/.ssh
rm known_hosts
ln -s /dev/null known_hosts
\end{verbatim}

\subsection{logs}
Logs will be at \texttt{/home/pmes/pmes/logs/}. Tomcat logs are at \texttt{\$CATALINA\_HOME/logs/catalina.out}
\section{Deploy PMES}
\subsection{PMES Service}
The PMES service is deployed using tomcat7. To deploy PMES service copy \texttt{pmes.war} to webapps folder (usually at \texttt{/var/lib/tomcat7/webapps}) and restart tomcat.

\begin{verbatim}
 sudo service tomcat7 stop
 sudo cp -r pmes.war /var/lib/tomcat7/webapps/
 sudo service tomcat7 start
\end{verbatim}


\subsection{Dashboard}
The Dashboard service is deployed using pm2.\\
\begin{verbatim}
 cd /home/pmes/pmes/Dashboard/PMES2Dash/
 # Install Dependencies
 npm install --save
 # Init pm2 Dashboard service
 pm2 init /home/pmes/pmes/Dashboard/PMES2Dash/pm.yaml
\end{verbatim}

Data is stored at mongodb database \texttt{pmes2}.
\begin{verbatim}
 # start mongo service
 sudo service mongodb start
 # Open mongo console
 mongo
 # Inside mongo console use pmes2 database
 > use pmes2
 # show pmes collections
 > show collections
 # show pmes users
 db.users.find()
 
\end{verbatim}

\section{Usage}
\subsection{Dashboard}
PMES Dashboard is  deployed using pm2. The endpoint is \texttt{http://localhost:3000}. There is an initial user created \texttt{user: pmes@pmes.com, password: pmes}

\subsection{PMES Service}
PMES service is  deployed using tomcat7. The endpoint is \texttt{http://localhost:8080/pmes/pmes/}.

You can call the service using curl (see APIDefinition document) or you can call the service using the python script \texttt{/home/pmes/pmes/scripts/curlPmesApi.py}.

\begin{verbatim}
 # api call getSystemStatus
 python3 curlPmesApi.py getSystemStatus
 # api call getActivityReport
 python3 curlPmesApi.py getActivityReport job_id
 # api call terminateActivity
 python3 curlPmesApi.py terminateActivity job_id
 # api call createActivity
 python3 curlPmesApi.py createActivity createVM.json
\end{verbatim}

\texttt{createVM.json} is a json file with a job definition. For example:
\begin{verbatim}
 [{ "jobName": "HelloTest2_584817558cb7550b5e9970b0",
           "wallTime": "5",
           "minimumVMs": "1",
           "maximumVMs": "1",
           "limitVMs": "1",
           "initialVMs": "1",
           "memory": "1.0",
           "cores": "1",
           "disk" : "1.0",
           "inputPaths": ["/home/"],
           "outputPaths": ["/home/"],
           "mountPath":"",
           "numNodes": "1",
           "user":
              { "username": "lcodo",
                "credentials":
                { "pem": "/home/pmes/certs/pmes.pem",
                  "key": "/home/pmes/certs/pmes.key"}
              },
           "img": 
              { "imageName": "os_tpl#4f916ede-218b-47e4-93aa-b795a5acf813", 
                "imageType": "resource_tpl#721112dd-2f33-40eb-8975-7bd34dbabfc8"
                "cores": "2"
                "memory": "2.0"
                "disk": "4.0"
              },
           "app":
              { "name": "HelloTest2",
                "target": "/home/pmes/testSimple",
                "source": "launch.sh",
                "args": { "val1": "Hola", "val2": "Mundo" } ,
                "type": "COMPSs"
              },
           "compss_flags": {}
        }]
\end{verbatim}

\section{Dependencies}
The image and the template should have the following permissions: Use and Manage for user, group and other.
\subsection{PMES VM}
Dependencies:
\begin{itemize}

 \item Rocci Client - \verb|{https://github.com/gwdg/rOCCI-cli; https://rvm.io/rvm/install#explained}|
 \begin{verbatim}
  # Install occi client
  curl -L http://go.egi.eu/fedcloud.ui | /bin/bash -
 \end{verbatim}
 
 \subitem If occi uses certificates move grid-security certificates to /etc/
 \item tomcat7: install tomcat7 \texttt{sudo apt-get install tomcat7}
 \subitem be sure that tomcat7 is using java8. (Export java home)
  \begin{verbatim}
  sudo nano /usr/share/tomcat7/bin/setenv.sh
  export JAVA_HOME=/usr/lib/jvm/java-8-oracle/
 \end{verbatim}
 \subitem if default tomcat7 user is used: no extra configuration is needed.
 
 \subitem if the tomcat user is changed to pmes: the following configuration is needed.
 \begin{verbatim}
     sudo nano /etc/default/tomcat7 # change TOMCAT7_USER=pmes, TOMCAT7_GROUP=pmes
     
     sudo nano /etc/init.d/tomcat7 # change TOMCAT7_USER=pmes, TOMCAT7_GROUP=pmes
 \end{verbatim}
 \item mongodb: https://docs.mongodb.com/manual/tutorial/install-mongodb-on-ubuntu/
 \item pm2
 \item node (version $>= 0.8$)
 
 \end{itemize}

 \subsection{APP VM}
 Dependencies:
 \begin{enumerate}
  \item COMPSs
  \item cloud-init: http://cloudinit.readthedocs.io/en/latest/topics/examples.html
  \item package nis or cifs to mount shared folders. (see document mountFolders)
 \end{enumerate}

\section{Actual Deploys}
\subsection{COMPSs VM at bsccv02 - old cluster}
Template 105 has Ubuntu 14, COMPSs 2.0 and CIFS.\\
Template 104 has Ubuntu 16, COMPSs 2.0 and CIFS.

\subsection{PMES VM at bsccv02 - old cluster}
Dashboard is at http://192.168.122.12:3000/ \\
PMES Service is at http://192.168.122.12:8080/ \\

\subsection{EBI}
PMES service is at \texttt{http://localhost:8080/pmes/pmes/}. The access is explained at EBIDeployment document.

\end{document}
