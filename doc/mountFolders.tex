\documentclass[a4paper,10pt]{article}
\usepackage[utf8]{inputenc}
\usepackage{subcaption}

%opening
\title{Mount Shared Folders}
\author{BSC - Workflows and Distributed Computing}

\begin{document}

\maketitle

\section{Mount folders using NIS}
\subsection{VM Configuration}
\begin{enumerate}

\item Install NIS and automount:
\begin{verbatim}
sudo su
apt-get update
apt-get install portmap nis
apt-get install autofs
\end{verbatim}
\begin{enumerate}
\item be sure that \texttt{nisdomainname} is \texttt{mmb.pcb.ub.es}
\begin{verbatim}
dpkg-reconfigure nis
\end{verbatim}
\end{enumerate}

\item Configure NIS:
\begin{enumerate}
\item add the following lines to \texttt{/etc/yp.conf}
\begin{verbatim}
domain mmb.pcb.ub.es server 10.4.96.183
domain mmb.pcb.ub.es server 10.4.96.184
domain mmb.pcb.ub.es server 10.4.96.181
domain mmb.pcb.ub.es server 10.4.96.182
\end{verbatim}
\item adapt \texttt{/etc/nsswitch.conf}
\begin{verbatim}
# /etc/nsswitch.conf
#
# Example configuration of GNU Name Service Switch functionality.
# If you have the `glibc-doc-reference' and `info' packages installed, try:
# `info libc "Name Service Switch"' for information about this file.
passwd:         files nis
group:          files nis
shadow:         files nis
gshadow:        files

hosts:          files dns nis
networks:       files

protocols:      db files
services:       db files
ethers:         db files
rpc:            db files

netgroup:       nis

automount:      nis
\end{verbatim}
\end{enumerate}

\item Configure automount:
\begin{enumerate}
 \item create \texttt{/etc/auto.services} with the following line:
\begin{verbatim}
+auto.services
\end{verbatim}
\item create \texttt{/etc/auto.homes} with the following line:
\begin{verbatim}
+auto.homes
\end{verbatim}
\end{enumerate}

\item \texttt{systemctl add-wants multi-user.target rpcbind.service}

\item Restart services:
\begin{verbatim}
/etc/init.d/nis restart
/etc/init.d/autofs restart
\end{verbatim}

\end{enumerate}

\subsection{Write on mounted folders}
\subsubsection{Home Directories}
We can see where the home directories are using \texttt{ypcat passwd}. To have a home directory we need to create a link at \texttt{/usr/people/\$username} \\

The MMB homes are at:
\texttt{/orozco/homes/desktops/lcodo}

The BSC homes are at:
\texttt{/orozco/homes/servers/bsc/\$username}

\subsubsection{Write}
We can create a new user with the same \texttt{UID} and \texttt{GID}. Edit \texttt{/etc/passwd} and create the new user.\\

Or we can use the user to write \texttt{su \$user}

\section{Mount folders using CIFS}
\subsection{VM Configuration}
\begin{enumerate}
\item Install CIFS:
\begin{verbatim}
sudo su
apt-get install cifs-utils
\end{verbatim}
Configure cifs:

 \item Create file \texttt{/etc/modprobe.d/cifs}
 \begin{verbatim}
  options cifs CIFSMaxBufSize=130048
  options cifs cifs_min_rcv=64
 \end{verbatim}
 \item Execute \texttt{modprobe -r cifs}
 \item Mount directory 
 \begin{verbatim}
  [ -d /transplant ] || mkdir -p /transplant
  mount -t cifs //192.168.122.253/INBTransplant /transplant -o user=guest,
  password=guestTransplant01,rsize=130048,sec=ntlmssp
 \end{verbatim}
\end{enumerate}
\subsection{Write on mounted folder}
\subsubsection{Home Directories}
The BSC homes are at:
\texttt{/data2/INB/transplant/\$username}
\subsubsection{Write}
Create a new user with the same \texttt{UID} and \texttt{GID}. Edit \texttt{/etc/passwd} and create the new user.\\
Add a soft link to the folder 
\begin{verbatim}
 sudo groupadd -g 306 transplant
 sudo useradd -m -d /home/$USERNAME -s /bin/bash --uid 306 --gid 306 -G root $USERNAME
 sudo mv /home/$USERNAME /tmp
 sudo ln -s /transplant/testUser/test/ /home/test
\end{verbatim}


\end{document}
